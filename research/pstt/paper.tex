\documentclass[11pt]{article}

\usepackage[margin=1in]{geometry}
\usepackage{microtype}
\usepackage{graphicx}
\usepackage{booktabs}
\usepackage{amsmath, amssymb}
\usepackage{enumitem}
\usepackage{hyperref}
\usepackage{natbib}

\hypersetup{
  colorlinks=true,
  linkcolor=blue,
  citecolor=blue,
  urlcolor=blue
}

\newcommand{\system}{PSTT}
\newcommand{\SystemName}{Prosody-Synchronous Trust-Preserving Translation}

\title{\SystemName{} (\system): A Contract for Flow-First Conversational Mediation}

% TODO: Replace author block later (or use an ACL/CHI template).
\author{
Anonymous Author(s)\\
}

\date{\today}

\begin{document}
\maketitle

\begin{abstract}
We propose \SystemName{} (\system), a \emph{flow-first} speech-to-speech mediation framework for informal multilingual conversation.
Unlike conventional translation systems optimized primarily for semantic fidelity, \system treats conversation as a composite of (i) propositional content, (ii) prosodic timing and turn-taking structure, and (iii) paralinguistic signals such as laughter, fillers, and hesitation.
We formalize a \emph{conversational mediation contract}: a single governing principle, a failure taxonomy that distinguishes human-like noise from trust-breaking errors, and guardrails that prohibit predictive commitments in high-risk semantic regions (negation, quantities, names, commitments).
We further provide a feasibility map that decomposes the contract into known components (streaming ASR, incremental/simultaneous MT, prosody-conditioned synthesis, paralinguistic detection, confidence estimation) and outline evaluation protocols (including Wizard-of-Oz studies) focused on \emph{trust} and \emph{prosodic synchrony} rather than verbatim adequacy alone.
\end{abstract}

\section{Introduction}
Speech translation has rapidly advanced from cascaded pipelines to end-to-end models, and from offline translation to low-latency \emph{simultaneous} settings \citep{sperber2020speech, ma-etal-2019-stacl, ren-etal-2020-simulspeech}.
Yet everyday conversation is not merely a sequence of semantic propositions.
Humans manage turn-taking with fine temporal precision \citep{Sacks1974a}, use disfluencies and self-repair as normal interactional tools \citep{Shriberg2001, schegloff1977preference}, and rely on paralinguistic cues (laughter, fillers, hesitation) to regulate rapport and intent \citep{Truong2007, Kaushik2015Laughter}.

This paper frames a distinct target: \emph{trusted, informal, low-stakes multilingual conversation} where maintaining perceived human presence and conversational rhythm is often more important than verbatim completeness.
In this regime, the most damaging failures are not minor paraphrases, but \emph{trust-breaking} errors: polarity flips (negation), fabricated specifics, false commitments, and speaker/intent misattribution.

\paragraph{Contributions.}
We contribute:
\begin{itemize}[leftmargin=*, itemsep=0.25em]
  \item A \textbf{conversational mediation contract} for flow-first real-time speech-to-speech mediation, including a signal model, a governing objective, and a failure taxonomy (acceptable vs.\ forbidden failures).
  \item A set of \textbf{predictive guardrails} specifying when early generation is permitted and when it must be delayed, hedged, or de-specified.
  \item A \textbf{feasibility map} connecting contract requirements to known technical methods in streaming ASR/MT/TTS, paralinguistic detection, and confidence estimation \citep{graves2012sequence, pmlr-v70-raffel17a, ma-etal-2019-stacl, li2021confidence}.
  \item An \textbf{evaluation framework} emphasizing prosodic synchrony and trust, including Wizard-of-Oz protocols \citep{dahlback1993wizard, kelley1984iterative}.
\end{itemize}

\section{Problem Setting and Scope}
\subsection{What \system is (and is not)}
\system is not a general-purpose translator.
It is a \emph{personality-preserving conversational mediator} for \textbf{trusted, informal} interactions where participants can tolerate small drift but not trust violations.
We explicitly exclude high-stakes domains (legal, medical, financial, contractual) where even ``human-like'' uncertainty is unacceptable.

\subsection{Interaction and channel model}
We assume a dyadic (or small-group) setting where each participant speaks naturally in their own language, and receives mediated speech in that same language from the other side(s).
In typical usage, each participant need not hear the target-language audio produced for their interlocutor; the system is therefore free to optimize the outgoing channel for the listener while keeping the speaker experience natural (e.g., via local audio routing or earbud playback).
Any ``traceability'' UI (e.g., confidence indicators, flags) should be off-channel and optional, to avoid breaking conversational immersion.

\subsection{Explicit non-goals}
\system does \emph{not} guarantee verbatim translation, legal or factual precision, or suitability for official contexts.
Conversational self-repair and mild vagueness are expected behaviors under the contract, not failures.

\subsection{Operating constraints}
\paragraph{Real time means prosodic synchrony.}
We treat ``real time'' as maintaining turn-taking rhythm and overlap behavior, not merely minimizing end-to-end latency.
The system should enter the channel at turn start, avoid unnatural silence, and preserve the timing of transition-relevance places (TRPs) \citep{Sacks1974a}.

\paragraph{Selective slowdown is allowed.}
Delays are acceptable \emph{locally} in high-risk semantic zones (negation, numbers, names, commitments), provided the interaction remains natural (e.g., through hedging, fillers, or partial framing).

\section{Related Work}
\subsection{Simultaneous and incremental translation}
Simultaneous MT formalizes the quality--latency trade-off and motivates policies that interleave reading and writing \citep{gu2017learning, ma-etal-2019-stacl}.
Streaming speech translation extends these ideas to speech, often using segmentation and wait-$k$-style constraints \citep{ren-etal-2020-simulspeech}.
Monotonic attention provides a differentiable mechanism for online alignment \citep{pmlr-v70-raffel17a}.

\subsection{Speech-to-speech translation}
Direct speech-to-speech translation has been demonstrated in end-to-end settings \citep{jia2019direct, jia2021translatotron2}.
These systems highlight feasibility, but are typically evaluated with text-centric metrics (e.g., BLEU) and often do not explicitly optimize for conversational trust and turn-taking naturalness.

\subsection{Conversation structure: turn-taking and repair}
Turn-taking is a core organizational principle of conversation \citep{Sacks1974a}.
Self-repair is not an error mode to eliminate but a structured interactional mechanism \citep{schegloff1977preference}; disfluencies can be informative and context-sensitive \citep{Shriberg2001}.
Modern dialogue systems model turn timing via continuous predictors and transformer-based models \citep{Skantze2017, ekstedt-skantze-2020-turngpt}.

\subsection{Paralinguistic events and confidence}
Automatic discrimination of laughter from speech is well studied \citep{Truong2007}, and joint laughter/filler detection has been explored in naturalistic audio \citep{Kaushik2015Laughter}.
For guardrails, confidence estimation in ASR is crucial to prevent premature commitments \citep{li2021confidence}.

\subsection{Interpreting theory and Wizard-of-Oz prototyping}
Conference interpreting research emphasizes cognitive load and trade-offs between accuracy and timing \citep{gile2009basic}.
Wizard-of-Oz methods are a pragmatic way to study interactional demands before full automation \citep{dahlback1993wizard, kelley1984iterative}.

\section{A Conversational Mediation Contract}
We formalize \system as a contract that constrains behavior in real time.
The contract is intended to be implementable and falsifiable: every observed failure must be classifiable under the taxonomy below.

\subsection{Governing principle}
\begin{quote}
\textbf{When forced to choose, \system must favor conversational flow and perceived human presence over semantic completeness, provided no hard semantic violations occur.}
\end{quote}

This principle shifts optimization from verbatim adequacy to \emph{safe mediation}: preserve rhythm and affect, but never fabricate commitments or polarity.

\subsection{Signal model}
We treat live speech as a layered signal:
\begin{enumerate}[leftmargin=*, itemsep=0.25em]
  \item \textbf{Linguistic content}: propositional meaning, requests, claims.
  \item \textbf{Prosodic structure}: timing, rhythm, intonation, overlap.
  \item \textbf{Paralinguistics}: laughter, sighs, fillers, hesitations, backchannels.
  \item \textbf{Meta-conversational events}: self-repair, rephrasing, clarification.
\end{enumerate}

\paragraph{Translation rule.}
Only linguistic content is translated; prosody and paralinguistics are preserved or mirrored; meta-conversational events are rendered naturally in the target language (not explained).

\subsection{Prosodic lock-in}
``Real time'' is operationalized as \emph{prosodic synchrony}: the mediated voice should begin promptly, track rhythm, and respect TRPs.
The system is allowed to use non-lexical vocalizations (e.g., target-language-appropriate fillers or backchannels) to maintain presence when semantic content is unstable.

\subsection{Failure taxonomy}
We define three failure classes (Table~\ref{tab:failure-taxonomy}).

\begin{table}[t]
\centering
\small
\begin{tabular}{@{}p{0.16\linewidth}p{0.78\linewidth}@{}}
\toprule
\textbf{Class} & \textbf{Description / examples} \\
\midrule
A (acceptable) &
Natural conversational noise: mild paraphrase drift, vagueness, self-repair markers, brief rephrasing. \\
\addlinespace
B (tolerable) &
Managed disruption: local slowdown in detail-heavy segments, hedging, partial framing that recovers naturally. \\
\addlinespace
C (forbidden) &
Trust-breaking errors: negation/polarity flips, fabricated specifics, false commitments, identity or speaker misattribution, emotional polarity inversion in sensitive contexts. \\
\bottomrule
\end{tabular}
\caption{Failure taxonomy for \system. The contract permits A, minimizes B, and forbids C.}
\label{tab:failure-taxonomy}
\end{table}

\subsection{Prediction contract (guardrails)}
Simultaneous systems often benefit from prediction/anticipation \citep{ma-etal-2019-stacl}.
\system permits prediction only when error impact is low.

\paragraph{Never predict (hard guardrails).}
\begin{itemize}[leftmargin=*, itemsep=0.25em]
  \item Negation and polarity markers (e.g., ``not'', ``never'')
  \item Numbers, quantities, dates, times
  \item Names and proper nouns (entities)
  \item Commitments/approvals (``yes'', ``no'', ``I agree'', ``I will'')
  \item Sensitive-domain content (health, money, legal commitments)
\end{itemize}
If these are unstable, the system must hold, hedge, or ask for clarification.
Guardrails can be implemented via token- and segment-level confidence, NER, and risk classifiers \citep{lample2016neural, li2021confidence, morante2012modality}.

\paragraph{Conditionally predict.}
Discourse markers, politeness strategies, and clause framing can often be produced in a way that remains compatible with multiple future completions.

\paragraph{Freely predict.}
Prosodic timing, backchannels, and non-lexical ``presence'' cues can be produced early, since they rarely create Class C failures.

\subsection{Translation gating for non-verbals}
When laughter, dominant fillers, or hesitation are detected, translation output should be gated: no propositional content is emitted, but paralinguistic mirroring and timing continuity continue \citep{Truong2007, Kaushik2015Laughter}.
This prevents ``over-translating humanity'' while preserving affect and conversational flow.

\subsection{Context-sensitive conservatism}
\system adapts conservatism based on conversational context:
\begin{itemize}[leftmargin=*, itemsep=0.25em]
  \item \textbf{Flow zones} (small talk, storytelling, low-specificity opinions): aggressive flow preservation, optimistic mediation.
  \item \textbf{Detail zones} (names, numbers, logistics): selective slowdown and near-zero guessing.
  \item \textbf{Sensitive zones} (health, money, family conflict, politics): expanded guardrails and tone neutrality.
\end{itemize}
This can be implemented as a risk-aware decoding policy conditioned on topic and confidence signals.

\subsection{Repair and recovery semantics}
\paragraph{Repair.}
Repairs must be additive or reframing, consistent with the conversational preference for self-correction \citep{schegloff1977preference}.
No apologies or system explanations occur in the mediated speech channel.

\paragraph{Recovery.}
After a rare hard correction, the system should revert to the last stable semantic state and re-enter with a clean utterance, prioritizing re-immersion over transparency.

\section{Feasibility Map}
The contract is only useful if each requirement maps to at least one plausible implementation technique.
Table~\ref{tab:feasibility} summarizes a non-exhaustive mapping; \system's novelty lies primarily in \emph{coordinating} these components under the failure taxonomy, rather than inventing new primitives.

\begin{table}[t]
\centering
\small
\begin{tabular}{@{}p{0.32\linewidth}p{0.62\linewidth}@{}}
\toprule
\textbf{Contract requirement} & \textbf{Plausible methods (examples)} \\
\midrule
Prosodic lock-in (early turn entry, TRP timing) &
Streaming ASR with partial hypotheses \citep{graves2012sequence}; turn-taking predictors \citep{Skantze2017, ekstedt-skantze-2020-turngpt}; low-latency synthesis. \\
\addlinespace
Signal separation (speech vs laughter/fillers) &
Paralinguistic classifiers and laughter/filler detection \citep{Truong2007, Kaushik2015Laughter}. \\
\addlinespace
Never-predict guardrails (negation, entities, commitments) &
Confidence estimation \citep{li2021confidence}; NER \citep{lample2016neural}; negation/modality modeling \citep{morante2012modality}. \\
\addlinespace
Local holding / hedging while speaking &
SimulMT policies and incremental decoding \citep{gu2017learning, ma-etal-2019-stacl}. \\
\addlinespace
Human-like repair (additive, no system meta-talk) &
Conversation-analytic repair structure \citep{schegloff1977preference}; incremental regeneration of recent clauses. \\
\bottomrule
\end{tabular}
\caption{Feasibility map: each \system requirement maps to known methods.}
\label{tab:feasibility}
\end{table}

\section{Architecture Sketch}
We outline a minimal architecture consistent with the contract (Figure~\ref{fig:architecture}).

\begin{figure}[t]
\centering
\fbox{\parbox{0.95\linewidth}{
\small
\textbf{Streaming ASR} $\rightarrow$ \textbf{Incremental MT} $\rightarrow$ \textbf{Commit / Guardrail Manager} $\rightarrow$ \textbf{Prosody \& Paralinguistic Planner} $\rightarrow$ \textbf{Low-latency TTS} \\
\quad\quad\quad\quad\quad\quad\quad\quad\quad\quad\quad\quad
\textbf{Paralinguistic / VAD classifiers} $\rightarrow$ gating signals
}}
\caption{A contract-aligned architecture sketch. Each block can be implemented with existing techniques; novelty lies in the policy logic that enforces the failure taxonomy.}
\label{fig:architecture}
\end{figure}

\subsection{Streaming ASR}
Streaming ASR provides partial hypotheses with token-level uncertainty.
RNN-Transducers \citep{graves2012sequence} and related streaming architectures enable low-latency decoding, which is a prerequisite for prosodic lock-in.

\subsection{Incremental MT / SimulMT}
Incremental MT can be implemented via fixed-latency policies such as wait-$k$ \citep{ma-etal-2019-stacl} or adaptive read/write policies \citep{gu2017learning}.
Online alignment mechanisms, including monotonic attention, can reduce recomputation and support incremental commitment \citep{pmlr-v70-raffel17a}.

\subsection{Commit manager and guardrails}
The commit manager decides which target tokens are safe to emit now versus which must be delayed, hedged, or de-specified.
Inputs include:
\begin{itemize}[leftmargin=*, itemsep=0.25em]
  \item token/segment confidence (ASR and MT),
  \item entity and number detection,
  \item negation risk markers,
  \item domain/sensitivity classifiers,
  \item dialogue state (last stable semantic state).
\end{itemize}

\subsection{Prosody and paralinguistic planner}
Prosody and paralinguistics are first-class outputs.
When propositional content is gated, the system can still produce appropriate presence cues (e.g., acknowledgments, hesitation markers) to preserve synchrony without risking Class C failures.

\section{Evaluation Framework}
\subsection{What to measure}
We argue for evaluation targets beyond adequacy/fluency:
\begin{itemize}[leftmargin=*, itemsep=0.25em]
  \item \textbf{Class C rate}: frequency of forbidden failures (polarity flips, false commitments, fabricated specifics, misattribution).
  \item \textbf{Prosodic synchrony}: deviation in onset timing and TRP alignment; overlap/gap statistics relative to human baselines \citep{Sacks1974a}.
  \item \textbf{Conversational naturalness}: subjective ratings of ``human presence'' and ``flow''.
  \item \textbf{Recovery quality}: how quickly interaction returns to smooth flow after disruptions.
\end{itemize}

\subsection{Protocols}
\paragraph{Wizard-of-Oz (WoZ).}
WoZ studies can validate the contract and collect interaction data before full automation \citep{dahlback1993wizard, kelley1984iterative}.
A human mediator can follow the contract rules (including guardrails) to establish human tolerance thresholds for hedging, delays, and repairs.

\paragraph{Incremental technical MVP.}
After WoZ, a narrow technical MVP (single language pair, limited domains) can test whether automatic components can satisfy the Class C constraints under realistic latency.

\section{Limitations and Ethical Considerations}
The contract explicitly excludes high-stakes domains.
Even in low-stakes contexts, voice preservation in direct speech-to-speech translation introduces misuse risks, and prior work has proposed mitigations \citep{jia2021translatotron2}.
We emphasize that the goal is trusted informal communication, not authoritative translation.

\section{Conclusion}
We presented \system, a contract-driven framing for flow-first conversational mediation.
By separating acceptable conversational noise from forbidden trust-breaking failures and by formalizing guardrails for prediction, \system offers a practical target for building real-time multilingual mediation that feels human without pretending to be perfect.

\bibliographystyle{plainnat}
\bibliography{references}

\end{document}
