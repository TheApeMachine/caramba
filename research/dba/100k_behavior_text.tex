% Suggested text for behavioral results section

To evaluate whether DBA's architectural compression affects model behavior, we constructed
an extended behavioral probe suite spanning 15 cognitive categories with 117 test cases,
including exact copy tasks, few-shot learning, distractor filtering, logical reasoning,
arithmetic, sequence completion, world knowledge, semantic understanding, format preservation,
long-context retrieval, robustness to rephrasing, edge cases, nuanced attention patterns,
instruction following, and consistency across equivalent prompts.

Results demonstrate \textbf{behavioral parity} between DBA and baseline (Table~\ref{tab:behavior-summary}).
Overall accuracy is nearly identical (27.4\% vs 26.5\%), with DBA passing 32 tests compared
to baseline's 31. In head-to-head comparison on tests where the models diverge, DBA wins
7 to baseline's 6. Category-level analysis shows each architecture winning 2 categories
with 11 ties.

Notably, DBA shows improved performance on reasoning tasks (+20\% absolute) and world
knowledge (+16.7\%), while baseline performs better on distractor filtering ($-$12.5\%)
and exact copy tasks ($-$14.3\%). The copy task deficit appears related to DBA's tendency
toward continuation rather than termination---a generation behavior rather than a
representational limitation.

These results support our central claim: \textbf{DBA achieves 37.5\% KV-cache reduction
while preserving behavioral capabilities}. The architectural bottleneck successfully
compresses the attention representation without degrading the model's ability to perform
diverse cognitive tasks.
